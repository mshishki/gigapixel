Recent trends in the semiconductor technologies outline the possibility to acquire images with the resolution surpassing the physical limits of human perception. Resulting images lie in the gigapixel range; compared with the modern consumer electronics, the resolution of such sensors is higher by three orders of magnitude. 

Here, we explore a recent gigapixel imaging solution, the specialized sensor model with a  spectral sensitivity response and the spatial resolution nearing that of a photographic film. The pixels incorporated on an array converge to the theoretical minimum size limit -- to the point where their respective full well capacitance is limited to a photon each -- thus providing a non-linear binary response with the intensity distribution that adheres to the fundamental physical principles. To accommodate for the diffraction limits of optics, the sensor values are subjected to oversampling; temporal oversampling is possible as well to achieve better estimation of factual intensity values.

We present the image formation model of the quanta imaging sensor which should help simulate image capture in a sub-diffraction resolving light field. Later, we report on the most commonly advised method used for the image reconstruction and the parameters needed for accurate estimation with present noise sources. Due to the compact design of the photoelements, quanta imaging sensor provides high readout speeds with high resolution and inherits most of the advantages of the current state-of-the-art CMOS.


